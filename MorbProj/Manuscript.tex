% manuscript, to be drafted

%%This is a very basic article template.
%%There is just one section and two subsections.
%\documentclass[12pt,oneside,a4paper,doublespacing]{article} % for submission
\documentclass[11pt,oneside,a4paper]{article} % for sharing

\usepackage{appendix}
\usepackage{amsmath}
\usepackage{caption}
\usepackage{placeins}
\usepackage{graphicx}
\usepackage{subcaption}
%\usepackage{subfig}
\usepackage{longtable}
\usepackage{setspace}
%\usepackage{tikz}
\usepackage{booktabs}
\usepackage{tabularx}
\usepackage{xcolor,colortbl}
\usepackage{chngpage}
%\usepackage[active,tightpage]{preview}
\usepackage{natbib}
\bibpunct{(}{)}{,}{a}{}{;} 
\usepackage{url}
\usepackage{nth}
\usepackage{authblk}
\usepackage[most]{tcolorbox}
%\usepackage{hyperref}
%\usepackage{color}
%\usepackage{fontspec}
%\usepackage{pdfsync}
\usepackage[normalem]{ulem}
\usepackage{amsfonts}
\renewcommand{\listtablename}{List of Appendix Tables}
\newcolumntype{C}[1]{>{\centering\let\newline\\\arraybackslash\hspace{0pt}}m{#1}}
\newcolumntype{L}[1]{>{\raggedright\let\newline\\\arraybackslash\hspace{0pt}}m{#1}}
% working on this need to concatenate file name based on sex and variable name
%\newcommand\Cell[1]{{\raisebox{-0.05in}{\includegraphics[height=.2in,width=.2in]{Figures/ColorCodes/\expandafter#1}}}}  

%%%%%%%%%%%%%%%%%%%%%%%%%%%%%%%%%%%%%%%%%%%%%%%%%%%%%%%%%%%%%%%%%%%%%%%%%%%%%
% setting color to letters affects spacing. Here's a hack I found here:
% http://tex.stackexchange.com/questions/212736/change-letter-colour-without-losing-letter-spacing
%\DeclareRobustCommand{\spacedallcaps}[1]{\MakeUppercase{\textsc{#1}}} % all
% caps with better spacing

%\colorlet{RED}{red}
%\colorlet{BLUE}{b}
\colorlet{rd}{red}
\colorlet{bl}{blue}

%%%%%%%%%%%%%%%%%%%%%%%%%%%%%%%%%%%%%%%%%%%%%%%%%%%%%%%%%%%%%%%%%%%%%%%%%%%%%%

\newcommand\ackn[1]{%
  \begingroup
  \renewcommand\thefootnote{}\footnote{#1}%
  \addtocounter{footnote}{-1}%
  \endgroup
}
\newcommand\vt[1]{\textcolor{rd}{#1}}
\newcommand\eg[1]{\textcolor{bl}{#1}}

\newcommand\tg[1]{\includegraphics[scale=.5]{Figures/triadtable/triad#1.pdf}}
\newcommand\tgh[1]{\raisebox{-.25\height}{\includegraphics[scale=.3]{Figures/triadtable/triad#1.pdf}}}

\defcitealias{HMD}{HMD}

% junk for longtable caption
\AtBeginEnvironment{longtable}{\linespread{1}\selectfont}
\setlength{\LTcapwidth}{\linewidth}

%%%%%%%%%%%%%%%%%%%%%%%%%%%%%%%
\begin{document}

\title{Accounting for temporal variation in morbidity projections}

\author[1]{Tim Riffe\thanks{riffe@demogr.mpg.de}}
\author[2]{Pil H. Chung}
\author[3]{John MacInnes}
\affil[1]{Max Planck Institute for Demographic Research}
\affil[2]{Department of Demography, University of California, Berkeley}
\affil[3]{School of Social and Political Science, University of Edinburgh}

%\author{[Authors]}

\maketitle

\begin{abstract}
This is important stuff!
\end{abstract}

Age standardization is an essential tool for the contemporary practice of
demography. Demographers age-standardize in order to assess trends and
intensities in rates that vary in regular ways over age free from distortion in
population structure. Without age standardization or its many cognates, we
would judge trends and magnitudes based on crude rates, which are now understood
not to carry the same predictive utility as rates that have been purged of
structure. This is a cornerstone tenet of contemporary demography. Typically, in
order to drive the point home, instructors find or concoct an example where a
comparison of age-standardized rates leads to the opposite conclusion as crude
rates suggest. This is quite motivating for the pupil, and it soon become second
nature. This is the point we wish to remake with respect to unaccounted-for
temporal variation in rates that are not vital rates.

Some processes vary over the life course, that is to say, within and over the
lives of individuals. If members of the same birth cohort are thought to have
something in common, it will surely be the case that members of the same birth
cohort that also end up dying in the same year share even more features in their
life course: Different aspects of their lives will on average align in
empirically regular ways. Likewise, persons dying in the same year probably
share many characteristics, especially if they die of some variety of intrisic
or unavoidable mortality.


\end{document}
