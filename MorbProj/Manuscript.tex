% manuscript, to be drafted

%%This is a very basic article template.
%%There is just one section and two subsections.
%\documentclass[12pt,oneside,a4paper,doublespacing]{article} % for submission
\documentclass[11pt,oneside,a4paper]{article} % for sharing

\usepackage{appendix}
\usepackage{amsmath}
\usepackage{caption}
\usepackage{placeins}
\usepackage{graphicx}
\usepackage{subcaption}
%\usepackage{subfig}
\usepackage{longtable}
\usepackage{setspace}
%\usepackage{tikz}
\usepackage{booktabs}
\usepackage{tabularx}
\usepackage{xcolor,colortbl}
\usepackage{chngpage}
%\usepackage[active,tightpage]{preview}
\usepackage{natbib}
\bibpunct{(}{)}{,}{a}{}{;} 
\usepackage{url}
\usepackage{nth}
\usepackage{authblk}
\usepackage[most]{tcolorbox}
%\usepackage{hyperref}
%\usepackage{color}
%\usepackage{fontspec}
%\usepackage{pdfsync}
\usepackage[normalem]{ulem}
\usepackage{amsfonts}
%\renewcommand{\listtablename}{List of Appendix Tables}
%\newcolumntype{C}[1]{>{\centering\let\newline\\\arraybackslash\hspace{0pt}}m{#1}}
%\newcolumntype{L}[1]{>{\raggedright\let\newline\\\arraybackslash\hspace{0pt}}m{#1}}
% working on this need to concatenate file name based on sex and variable name
%\newcommand\Cell[1]{{\raisebox{-0.05in}{\includegraphics[height=.2in,width=.2in]{Figures/ColorCodes/\expandafter#1}}}}  

%%%%%%%%%%%%%%%%%%%%%%%%%%%%%%%%%%%%%%%%%%%%%%%%%%%%%%%%%%%%%%%%%%%%%%%%%%%%%
% setting color to letters affects spacing. Here's a hack I found here:
% http://tex.stackexchange.com/questions/212736/change-letter-colour-without-losing-letter-spacing
%\DeclareRobustCommand{\spacedallcaps}[1]{\MakeUppercase{\textsc{#1}}} % all
% caps with better spacing

%\colorlet{RED}{red}
%\colorlet{BLUE}{b}
%\colorlet{rd}{red}
%\colorlet{bl}{blue}

%%%%%%%%%%%%%%%%%%%%%%%%%%%%%%%%%%%%%%%%%%%%%%%%%%%%%%%%%%%%%%%%%%%%%%%%%%%%%%

\newcommand\ackn[1]{%
  \begingroup
  \renewcommand\thefootnote{}\footnote{#1}%
  \addtocounter{footnote}{-1}%
  \endgroup
}
%\newcommand\vt[1]{\textcolor{rd}{#1}}
%\newcommand\eg[1]{\textcolor{bl}{#1}}

%\newcommand\tg[1]{\includegraphics[scale=.5]{Figures/triadtable/triad#1.pdf}}
%\newcommand\tgh[1]{\raisebox{-.25\height}{\includegraphics[scale=.3]{Figures/triadtable/triad#1.pdf}}}

\defcitealias{HMD}{HMD}
\newcommand{\dd}{\; \mathrm{d}}
\newcommand{\tc}{\quad\quad\text{,}}
\newcommand{\tp}{\quad\quad\text{.}}
% junk for longtable caption
\AtBeginEnvironment{longtable}{\linespread{1}\selectfont}
\setlength{\LTcapwidth}{\linewidth}

%%%%%%%%%%%%%%%%%%%%%%%%%%%%%%%
\begin{document}

\title{Accounting for temporal variation in morbidity measurement
and projections}

\author[1]{Tim Riffe\thanks{riffe@demogr.mpg.de}}
\author[2]{Pil H. Chung}
\author[3]{John MacInnes}
\affil[1]{Max Planck Institute for Demographic Research}
\affil[2]{Department of Demography, University of California, Berkeley}
\affil[3]{School of Social and Political Science, University of Edinburgh}

%\author{[Authors]}

\maketitle

\begin{abstract}
This is important stuff!
\end{abstract}


\section*{Introduction}
Age standardization is an essential tool for the contemporary practice of
demography. Demographers age-standardize in order to assess trends and
intensities in rates that vary in regular ways over age free from distortion in
population structure. Without age standardization or its many cognates, we
would judge trends and magnitudes based on crude rates, which are now understood
not to carry the same predictive utility as rates that have been purged of
structure. This is a cornerstone tenet of contemporary demography. Typically, in
order to drive the point home, instructors find or concoct an example where a
comparison of age-standardized rates leads to the opposite conclusion as crude
rates suggest. This is quite motivating for the pupil, and it soon become second
nature. This is the point we wish to remake with respect to unaccounted-for
temporal variation in rates that are not vital rates.

Some processes vary over the life course, that is to say, within and over the
lives of individuals. If members of the same birth cohort are thought to have
something in common, it will surely be the case that members of the same birth
cohort that also end up dying in the same year share even more features in their
life course: Different aspects of their lives will on average align in
empirically regular ways. In general, persons dying in the same year probably
share many characteristics in the time prior to death, especially persons that
die of intrisic or unavoidable mortality. This empirical
alignment will in some cases hold, even if the persons are not from proximate
birth cohorts. That is to say, for some conditions, temporal variation in terms
of remaining years of life or completed lifespan provides sharper and more
regular relief than variation in terms of chronological age. In such instances,
age standardization of the common variety does not provide the degree of
control or precision that is often assumed -- it is misapplied. By extension,
chronological age patterns of characteristics that do not vary primarily as a
function of chronological age are misleading and accidental. 

In this paper, we provide an empirical example of a health condition that
appears to have a chronological age pattern in the margin, but has none when
disagregated. We show how even short term projections of the
condition into the future imply great differences in prevalence--- changes not only of different magnitude but
of different sign--- with respect to a projection based only on an
apparent marginal age pattern. The empirical excercise is relatively modest,
given the available data, but we do demonstrate a vulnerability in the
methodological status quo that has consequences for how we measure health at the
population level. We hope that this basic observation will motivate the
development of new extensions of standardization methods, as the standardization
we propose is more amenable to ``clean'' count data that are not readily
available in most instances.

\section*{Motivation}
Some words from John

\section*{Formal relationships}
Say we want to measure a health condition, $G$, something bad and degenerative
that usually sets in late in life, but with onset spread out over a wide range
of ages. For simplicity, imagine that $G$ is a binary condition: either you have
it or you don't. The same excercise could be carried out with adjustments for
intensity, but this complicates the measurement of ``how much'' condition $G$
there is, and the implied methodlogical details might obfuscate the point we
wish to make. For a binary condition, a straightforward calculation is a
proportion of individuals with $G$. Say then that the number of people
with condition $G$ at chronological age $a$, $G(a)$, is simple the people in age
$a$ times the probability of having $G$ in that age, $g'(a)$:
\begin{equation}
\label{eq:Ga}
G(a) = P(a)  g'(a) \tp
\end{equation}
\eqref{eq:Ga} holds tautologically in the given year of observation, but it may
not hold in the future if the essential variation in $G$ in fact varies by
remaining years of life, $y$, one's thanatological age. In this case, $g'(a)$ is
itself an aggregate that depends on the real underlying thantological function
of $G$, $g(y)$, and the population structure classified by both chronolgical and
thanatological age:
\begin{align}
\label{eq:gpa}
g'(a) =& \frac{\int _0^\omega g(y)  P(y,a) \dd y}{P(a)} \\
  =& \frac{\int _0^\omega g(y)  P(a) \mu(a+y)\frac{l(a+y)}{l(a)}\dd y}{P(a)}\\
  =& \int _0^\omega g(y) \mu(a+y)\frac{l(a+y)}{l(a)}\dd y \\
  =& \int _0^\omega g(y) f(y|a)\dd y \tc
\end{align}
where $\omega$ is the highest age a person can survive to,
$\mu(a)$ is the force of mortality at age $a$, and $l(a)$ is the
survivor function at age $a$. \eqref{eq:gpa} means that $g'(a)$ is not
independent of mortality in the case that it is sufficiently described by
$g(y)$. This conclusion is rather intuitive in any case, since for a condition
to vary as a function of thanatological age means that it is bound up with
variation in lifespan itself. In this case, \eqref{eq:Ga} is better rewritten
as:
\begin{equation}
G(a) = P(a) \int _0^\omega g(y) f(y|a)\dd y \tp
\end{equation}
There may be a sort-of-empirically-regular pattern to $g'(a)$, but it is
accidental. $g'(a)$ will move more from year to year than will $g(y)$, simply
because mortality also changes. Likewise, the mix of ages in a
particular year are subject to different mortality schedules, as they belong to
different birth cohorts, each with its own history and future of mortality.


\end{document}
